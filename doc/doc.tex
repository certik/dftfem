%&csplain

\input eplain   
\input epsf

%\voffset=-1.5cm
%\hoffset=-1.5cm
%\advance \vsize by 3cm
%\advance \hsize by 3cm

\magnification\magstep1
%\chyph
\csaccents
%\raggedbottom

\newcount\eqcount
\def\incno#1{\global\advance\eqcount1\definexref{#1}{\the\eqcount}{eq}}
\def\no#1{\incno{#1}\eqno{(\the\eqcount)}}
\def\nno#1{\incno{#1}&{(\the\eqcount)}}
\def\rno#1{(\refn{#1})}


\def\mydots{\leaders\hbox{$\mkern1.5mu.\mkern1.5mu$}\hfill}
\def\tocsecentry#1#2#3{\line{\bf#2 #1\thinspace\mydots#3}}
\def\tocsubentry#1#2#3{\line{\quad#2 #1\thinspace\mydots#3}}

\newcount\secno
\newcount\subno
\def\secskip{\vskip 18pt plus 6pt minus 6pt}

\font\titlefont=\fontname\tenbf\space scaled\magstep2
\font\subfont=\fontname\tenbf\space scaled\magstep1
\font\subsubfont=\fontname\tenbf\space scaled\magstep0
\font\progfont=cstt8
\def\title#1{\centerline{\titlefont #1}\bigskip}
\def\sec#1\par{\advance\secno by1\subno=0\secskip\goodbreak\noindent%
{\subfont \the\secno. #1}%
\writenumberedtocentry{sec}{#1}{\the\secno.}\medskip\nobreak\noindent}
\def\sub#1\par{\advance\subno by 1\medskip\goodbreak\noindent%
{\edef\n{\the\secno.\the\subno}{\subsubfont\n\ #1}%
\writenumberedtocentry{sub}{#1}{\n}}\par\nobreak\noindent}

\newcount\figno
\def\dref#1{\global\advance\figno by 1\definexref{#1}{\the\figno}{}\the\figno}

\font\twelverm=csr12
\def\d{{\rm d}}

\def\psfig#1#2#3{\medskip\noindent\epsfxsize=\hsize\epsfbox{#1}
\centerline{Fig. \dref{#2}: #3}\medskip}

\def\y#1{(#1 let)}
\def\degrees{^\circ}
\def\p#1#2{ {\partial#1\over\partial#2} }
\def\pp#1#2{ {\partial^2#1\over\partial#2^2} }

%end of macro definitions

\title{DFT and FEM}

\centerline{Ond�ej �ert�k}
\centerline{April, 2006}

\vfill\eject

\title{Contents}

\readtocfile

\sec{Schr\"odinger Equation}

$$\left(-{\hbar^2\over2m}\nabla^2 + V\right)\psi=E\psi\,.$$

\sub{Weak Formulation}

We multiply both sides by a test function $v$
$$-\left({\hbar^2\over2m}\nabla^2\psi\right)v=(E-V)\psi v\,,$$
and integrate over the whole volume we are interested in
$$\int-\left({\hbar^2\over2m}\nabla^2\psi\right)v\,\d V=\int(E-V)\psi v\,\d
V\,,\no{1}$$
and using the vector identity
$$-\left(\nabla^2\psi)\right)v=\nabla \psi\cdot
\nabla v - \nabla\cdot\left((\nabla \psi)v\right),$$
we rewrite the left hand side of \rno{1}
$$\int{\hbar^2\over2m}\nabla\psi\cdot\nabla v\,\d V=\int(E-V)\psi v\,\d
V+\int{\hbar^2\over2m}\nabla\cdot\left((\nabla \psi)v\right)\,\d V\,,$$
now we apply Gauss Theorem
$$\int{\hbar^2\over2m}\nabla\psi\cdot\nabla v\,\d V=\int(E-V)\psi v\,\d
V+\oint{\hbar^2\over2m}(\nabla \psi)v\cdot{\bf n}\,\d S\,,$$
and rewriting $\nabla\psi\cdot{\bf n}\equiv{\d\psi\over\d n}$
$$\int{\hbar^2\over2m}\nabla\psi\cdot\nabla v\,\d V=\int(E-V)\psi v\,\d
V+\oint{\hbar^2\over2m}{\d\psi\over\d n}v\,\d S\,,\no{w}$$
which is the weak formulation. The problem reads: find a function $\psi$ such
that \rno{w} holds for every $v$.

\sub{Weak Formulation}

We arrived at the weak formulation:
$$\int{\hbar^2\over2m}\nabla\psi\cdot\nabla v\,\d V+ \int vV\psi\,\d V
=
\int E\psi v\,\d V + \oint{\hbar^2\over2m}{\d\psi\over\d n}v\,\d S\,.$$

\sub{Linear Equations}

Now we substitute $\phi_i$ for $v$ and expand $\psi=\sum q_j\phi_j$
$$\left(\int{\hbar^2\over2m}\nabla\phi_j\cdot\nabla\phi_i\,\d V+
\int\phi_iV\phi_j\,\d V\right)q_j
=
\left(\int E\phi_j\phi_i\,\d V\right)q_j
+\oint{\hbar^2\over2m}{\d\psi\over\d n}\phi_i\,\d S\,,$$
which can be written in a matrix form
$$\left(K_{ij}+V_{ij}\right)q_j=EM_{ij}q_j+F_i\,,$$
where
$$\eqalign{
V_{ij}&=\int\phi_iV\phi_j\,\d V\,,\cr
M_{ij}&=\int\phi_i\phi_j\,\d V\,,\cr
K_{ij}&={\hbar^2\over2m}\int\nabla\phi_i\cdot\nabla\phi_j\,\d V\,,\cr
F_i&={\hbar^2\over2m}\oint{\d\psi\over\d n}\phi_i\,\d S\,.\cr
}$$
Usually we set $F_i=0$.

\sub{Finite Elements}

We decompose the domain into elements, so we need to compute the integrals over
one finite element:
$$
K_{ij}^{E}=\int{\hbar^2\over2m}\nabla\phi_j\cdot\nabla\phi_i\,\d V^{E}\approx
\sum_{q=0}^{N_q-1}{\hbar^2\over2m}\,\nabla\phi_i(x_q)\cdot\nabla\phi_j(x_q)\,
w_q|\det J(\hat x_q)|\,.
$$
The surface integrals are computed accordingly.

%\sub{Boundary Conditions}
%
%The Neumann boundary condition is already taken into
%account in \rno{m} - it's the surface integral containing
%${\d u\over\d n}$, so we compute this integral over every element, where 
%${\d u\over\d n}\neq0$.
%
%The Dirichlet boundary condition is imposed by adding the following equation to
%\rno{d} for every element we want to prescribe the value of $u$:
%$$P\int\phi_i\phi_j\,\d Su_j=P\int u_0\,\phi_i\,\d S\,,$$
%where $P$ is some big number (penalty) and $u_0$ is the value of $u$ on the
%boundary. Then the matrices are (on this element) as follows:
%$$\eqalign{
%K_{ij}&=\int\lambda\nabla\phi_j\cdot\nabla\phi_i\,\d V+
%	P\int\phi_i\phi_j\,\d S\,,\cr
%F_i&=\int f\phi_i\,\d V+\oint\lambda{\d u\over\d n}\phi_i\,\d S+
%	P\int u_0\,\phi_i\,\d S\,.\cr
%}$$
%
%The second possibility to impose the Dirichlet boundary condition is to set the
%$i$ row in RHS to $u_0 P$ and the element $i,i$ in the stiffness matrix to
%$P$.
%
%The first way is more general and more precise because we don't bother about
%any nodes, we just need the value of $u_0(x,y,z)$ on the whole boundary (and
%take into account only the $u_0$'s in the Gauss points). The second way is
%better if we just want to set the value of $u$ in some particular nodes.
%

\sec{Atomic units (au)}

$\hbar=1$, mass of the electron $m=1$, charge of the electron $e=1$,
Bohr radius
$$a_0={4\pi\epsilon_0\hbar^2\over me^2}={\hbar\over mc\alpha}=1\,,$$
from which follows $4\pi\epsilon_0=1$ and $c\alpha=1$.
Hydrogen atom Hamiltonian in SI units:
$$H=-{\hbar^2\over2m}\nabla^2-{1\over4\pi\epsilon_0}{e^2\over r}$$
and au units:
$$H=-{1\over2}\nabla^2-{1\over r}\,.$$
The energy spectrum for hydrogen is
$$E_n=-{\hbar^2\over ma_0^2}{1\over2n^2}=-{1\over2n^2}\,.$$
Energy is measured in Hartrees, one Hartree is
$$E_h={\hbar^2\over ma_0^2}=mc^2\alpha^2=1=27.211\rm\,eV\,.$$
The fine-structure constant is dimensionless:
$$\alpha={1\over137.036}\,,$$
so the speed of light is $c={1\over\alpha}=137.036$.


\sec{Potential}

$$V=V_{ion} + V_H + V_{XC}\,,$$
where for 1 atom in the origin: 
$$V_{ion}=V_{loc}(r) + {\hat V}_{nonlocal}\,.$$
$V_{loc}(r)$, $V_H({\bf x})$ and $V_{XC}({\bf x})$ are scalar potentials,
${\hat V}_{nonlocal}$ is an operator. We can thus write:
$$V=V_{loc} + V_H + V_{XC} + {\hat V}_{nonlocal}\,.$$
Now let's look at the operator ${\hat V}_{nonlocal}$:
$$({\hat V}_{nonlocal}\psi_i)({\bf x})=\sum_l V_l {2l+1\over
4\pi}\int P_l({\bf\hat r\cdot\hat n})\psi_i(r{\bf\hat n})\,\d\Omega\,,$$
where 
$$\eqalign{
r&=|{\bf x}|\,,\cr
{\bf\hat r}&={{\bf x}\over|{\bf x}|}={{\bf x}\over r}\cr
}$$
and the integration is over a unit sphere, using the unit
vector ${\bf\hat n}$ as the integration variable. It
has these cartesian components as usual:
$${\bf\hat n}=(\sin\theta\cos\phi,\,\sin\theta\sin\phi,\,\cos\theta)\,,$$
in other words:
$$\int\d\Omega\, P_l({\bf\hat r\cdot\hat n})\psi_i(r{\bf\hat n})=
\int_0^{2\pi}\!\!\!\d\phi\int_0^{\pi}\!\!\d\theta\,
P_l({\bf\hat r\cdot\hat n})\psi_i(r{\bf\hat n})\sin^2\theta=$$
$$
\int_0^{2\pi}\!\!\!\!\!\!\!\!\d\phi\!\!\int_0^{\pi}\!\!\!\!\!\!\d\theta
P_l({x\over r}\sin\theta\cos\phi+{y\over r}\sin\theta\sin\phi+
{z\over r}\cos\theta)
\psi_i(r\sin\theta\cos\phi,r\sin\theta\sin\phi,r\cos\theta)\sin^2\theta\,.$$
Everything is known: $\phi_i(x,y,z)$ is a basis function, $P_l(x)$ is a
Legendre polynomial (defined for $0\le x\le 1$), first few are:
$$\eqalign{
P_0(x)&=1\,,\cr
P_1(x)&=x\,,\cr
P_2(x)&={1\over2}(3x^2-1)\,,\cr
P_3(x)&={1\over2}(5x^3-3x)\,,\cr
P_4(x)&={1\over8}(35x^4-30x^2+3)\,,\cr
P_5(x)&={1\over8}(63x^5-70x^3+15x)\,,\cr
P_6(x)&={1\over16}(231x^6-315x^4+105x^2-5)\,,\cr
}$$

\sec{Program}

\verbatim
for (int l=0;l<1;l+=1)
{
  Real I=0.0;
  for (double dphi=2*pi/3,phi=0; phi<2*pi; phi+=dphi)
  for (double dtheta=pi/3, theta=0; theta<pi; theta+=dtheta)
    I+=P(l,(x*sin(theta)*cos(phi)+y*sin(theta)*sin(phi)+
              z*cos(theta))/r)
        *phi_xyz(i,r*sin(theta)*cos(phi),r*sin(theta)*sin(phi),
              r*cos(theta)) 
        *pow(sin(theta),2)*dtheta*dphi;
  Vxpsi+=getVl(l,r)*I*(2*l+1)/(4*pi);
}
|endverbatim
\bigskip

The {\tt phi\_xyz(x,y,z)} function returns the $\phi_i({\bf x})=
\phi_i(x,y,z)$. The shape function is in fact defined on the reference element
as $\tilde\phi_i(\vec\xi)=\tilde\phi_i(\xi,\eta,\zeta)$ and 
$$\phi_i({\bf x})=\tilde\phi_i(\vec\xi({\bf x}))\,.$$
The dependence
$$\vec\xi=\vec\xi({\bf x})$$
has to be computed numerically using a Newton method.

%global coordinate system
%structure coordinate system
%overall coordinate system

%vs

%local coordinate system
%natural coordinate system
%reference plane/element

%isoparametric mapping

\sec{Test examples}

It is important to test the installation, both the assembly and the solvers.
For this reason there are several preprogrammed 3D examples.

Sometimes the spectrum depends on the geometry parameter $a$ (a size of a
box or a sphere). We suggest to compute in atomic units and first set $a=1$, later
to try other values of $a$ to check the correct dependence of the solution on
$a$.

\sub{Infinite square potential well}

We have a box of a side $a$ and a scalar potential:
$$V({\bf x})=\cases{0 &for ${\bf x}$ in the box,\cr
             \infty &otherwise.\cr}$$
Analytic solution gives the following spectrum:
$$E_{n_1n_2n_3}={{\hbar^2 \pi^2\over2ma^2}(n_1^2+n_2^2+n_3^2)}\,,
\qquad n_1,n_2,n_3=1,2,3,\dots\,,$$
so a few lowest energies are (in atomic units):
$$\eqalign{
E_{111}&={3\pi^2\over2a^2}={14.804\over a^2}\,,\cr
E_{112}=E_{121}=E_{211}&={6\pi^2\over2a^2}={29.609\over a^2}\,,\cr
E_{122}=E_{212}=E_{221}&={9\pi^2\over2a^2}={44.413\over a^2}\,,\cr
E_{113}=E_{131}=E_{311}&={11\pi^2\over2a^2}={54.283\over a^2}\,,\cr
E_{222}&={12\pi^2\over2a^2}={59.218\over a^2}\,,\cr
E_{123}=E_{132}=E_{213}=E_{231}=E_{312}=E_{321}&={14\pi^2\over2a^2}={69.087\over a^2}\,,\cr
E_{223}=E_{232}=E_{322}&={17\pi^2\over2a^2}={83.892\over a^2}\,,\cr
E_{114}=E_{141}=E_{411}&={18\pi^2\over2a^2}={88.826\over a^2}\,,\cr
E_{133}=E_{313}=E_{331}&={19\pi^2\over2a^2}={93.761\over a^2}\,,\cr
}$$

A sample result on a hexahedral box mesh with 68921 nodes
and 64000 elements, $a=1$:
\medskip
\noindent14.812018\hfil\break
29.654619 29.654499 29.654420\hfil\break
44.497216 44.497001 44.496770\hfil\break
54.493911 54.493777 54.493637\hfil\break
59.339523\hfil\break
69.336572 69.336349 69.336243 69.336161 69.336105 69.335893\hfil\break
84.178904 84.178856 84.178715\hfil\break
89.483205 89.482994 89.482875\hfil\break
94.175499 94.175389 94.175115\hfil\break


\sub{Infinite spherical potential well}

We have a sphere of a radius $a$ and a spherically symmetric potential:
$$V(r)=\cases{0 &for $r\le a$,\cr
             \infty &for $r > a$.\cr}$$
Analytic solution gives the following spectrum:
$$E_{nl}={\hbar^2 u_{nl}^2\over2ma^2}\,,\qquad n=1,2,3,\dots;
\quad l=0,1,2,\dots\,,$$
where $u_{nl}$ is the $n$th zero of the spherical Bessel function $j_l$:
$$\eqalign{
u_{10}&=3.142\,,\cr
u_{11}&=4.493\,,\cr
u_{12}&=5.763\,,\cr
u_{20}&=6.283\,,\cr
u_{13}&=6.988\,,\cr
u_{21}&=7.725\,,\cr
}$$
so we finally have (in atomic units):
$$\eqalign{
E_{10}&={ 4.936\over a^2}\,,\cr
E_{11}&={10.094\over a^2}\,,\cr
E_{12}&={16.606\over a^2}\,,\cr
E_{20}&={19.738\over a^2}\,,\cr
E_{13}&={24.416\over a^2}\,,\cr
E_{21}&={29.838\over a^2}\,.\cr
}$$
A degeneracy of the energy level $E_{nl}$ is $2l+1$.

A sample result on a tetrahedral spherical mesh with 159192 nodes
and 982363 elements, $a=1$:
\medskip
\noindent4.939043\hfil\break
10.114084 10.114074 10.114044\hfill\break
16.660649 16.660463 16.660140 16.660083 16.660015\hfill\break
19.812318\hfill\break
24.528641 24.528246 24.528119 24.527966 24.527687 24.527631 24.527461\hfill\break
30.007858 30.007381 30.006918\hfill\break
33.69104 33.69050 33.69005 33.68987 33.68953 33.68938 33.68913 33.68904
33.68880\hfill\break
41.682994\hfill\break
 
\sub{Spherical linear harmonic oscillator}

We have a sphere of a radius $a$ and a spherically symmetric potential:
$$V(r)=\cases{{1\over2}m\omega^2r^2 &for $r\le a$,\cr
             \infty &for $r > a$.\cr}$$
Analytic solution (in the limit $a\to\infty$) gives the following spectrum:
$$E_{nl}=\hbar\omega\left(2n+l+{3\over2}\right)\,,\qquad n,l=0,1,2,3,\dots\,,$$
so a few lowest energies are (in atomic units and $\omega=1$):
$$\eqalign{
E_{00}&={3\over2}=1.5\,,\cr
E_{01}&=1+{3\over2}=2.5\,,\cr
E_{02}=E_{10}&=2+{3\over2}=3.5\,,\cr
E_{03}=E_{11}&=3+{3\over2}=4.5\,,\cr
E_{04}=E_{12}=E_{20}&=4+{3\over2}=5.5\,,\cr
}$$
A degeneracy of the energy level $E_{nl}$ is $2l+1$, so if we denote
$\Lambda=2n+l$, the spectrum can also (equivalently) be written as
$$E_{\Lambda}=\hbar\omega\left(\Lambda+{3\over2}\right)\,,\qquad \Lambda=
0,1,2,3,\dots$$
and the degeneracy of $E_{\Lambda}$ is ${1\over2}(\Lambda+1)(\Lambda+2)$,
a first few values are $1, 3, 6, 10, 15, \dots$

When computing numerically, we need to choose $a$ sufficiently large, and
create the mesh sufficiently dense around the origin to get correct results.
A sample result on a tetrahedral spherical mesh with 159149 nodes
and 982010 elements, $a=15$:

\medskip
\noindent1.537237\hfill\break
2.597473 2.593177 2.590595\hfill\break
3.680867 3.664781 3.663415 3.655095 3.653997 3.643068\hfill\break
4.82900 4.81065 4.79727 4.75939 4.75506 4.74968 4.74468 4.74315 
4.74157 4.73509\hfill\break
5.89124 5.87400 5.86998 5.86629 5.84858 5.84773 5.83902 5.83518
5.83029 5.82834 \dots\hfill\break

\sub{DFT computation of Si atom}

We have an Si atom in the origin, with the following pseudopotential:
$$V=V_{loc} + V_H + V_{XC} + {\hat V}_{nonlocal}\,,$$
$$({\hat V}_{nonlocal}\psi_i)({\bf x})=\sum_{l=0}^2 V_l {2l+1\over
4\pi}\int P_l({\bf\hat r\cdot\hat n})\psi_i(r{\bf\hat n})\,\d\Omega\,,$$
where $U(r)=V_{loc} + V_H + V_{XC}$ is given and is treated as a 
usual scalar potential, $V_0(r)$, $V_1(r)$ and $V_2(r)$ is also given,
see fig. \refn{pseudofig} for the plot of all these potentials in
atomic units. "rho" is the charge density from another DFT program.
\psfig{pseudopot.eps}{pseudofig}
{Scalar potential U and non-local potentials $V_l$}
I used a box (side $a=10$) divided into 30x30x30 equal hexahedrons (27000
elements, 29791 nodes), the atom is in the middle of the box, ie. there are at
least 5 atomic units of elements in every direction. Assembling took 10 minutes
or 66 minutes (depends if I only take into account the basis function on the
current element or not), solving depends on the number of eigenvalues, but
usually from 10 to 30 seconds. It was run on amd64 computer, which is around
5.5 faster than my 900Mhz Pentium III at home.

There weren't any differences when I used a
symmetric or general solver (!). I shifted the spectrum by $+10$. 

Result -- first case (the second column is the correct energy from another DFT program):

\verbatim
9.263832    9.60510
9.859375    9.84806016981541
9.859634
9.860282
10.055438   10.027909219877655
10.200252
10.200332
10.200356
10.251075
10.251122
|endverbatim
\bigskip
Result -- second case:

\verbatim
9.263867
9.855735
9.858280
9.859868
10.056769
10.199861
10.200029
10.200309
10.250798
10.250892
|endverbatim
\bigskip

I also computed a case with ${\hat V}_{nonlocal}=0$, the assembling took 10s.
Results:

\verbatim
9.267249
9.860491
9.860505
9.860518
10.056306
10.200382
10.200392
10.200400
10.251208
10.251215
|endverbatim
\bigskip

Differences are almost negligible. The mesh is probably too rough.

\bye
